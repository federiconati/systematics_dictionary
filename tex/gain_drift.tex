\subsection{Time variation}

\paragraph{Description:}
Depending on the technique used to calibrate the thermal-response of the timestream, there might be several science scans taken in between two calibration measurements. 
In between two calibration measurements, the thermal calibration of each detector might drift (\emph{time variation}) from its initial value, and significantly. Here sensitivity is defined at dP/dI and gain is defined as dT/dI or (dT/dP)*(dP/dI). Measurements of planets and known temperature sources like the POLARBEAR stimulator gives defines gain while studies of loopgain variation gives a tells your about sensitivity. With a fixed voltage bias on a TES in the strong electrothermal feedback regime the total power on the detector is fixed so the electric bias power will change to compensate for changes in optical power: $\Delta P_{opt}=-\Delta P_{bias}$. The TES Loopgain, $\mathcal{L}$, is linearly proportional to $P_{bias}$ so as the detector sees changes in atmospheric and instrumental loading $\mathcal{L}$ varies. Although we are looking at gain stability here, the bolometers thermal time constant is also sped up under electrothermal feedback by the relationship (1). So variations in loopgain will effect both sensitivity and the time constant over an observation.
\begin{equation}
\tau_{ETF}=\tau_0\frac{1+\beta_I+R_L/R_0}{1+\beta_I+R_L/R_0+(1-R_L/R_0)\mathcal{L}}
\end{equation}
$\tau_0$ is the intrinsic thermal time constant $G_0/C$. The power to current sensitivity, $s_I(\omega)$ is also dependent on $\mathcal{L}$ and is given by (2)
\begin{equation}
s_I(\omega)=\frac{-1}{I_0R_0}\left(\frac{L}{\tau_{elec}R_0\mathcal{L}}+(1-\frac{R_L}{R_0})+i\omega\frac{L\tau_0}{R_0\mathcal{L}}(\frac{1}{\tau_{ETF}}+\frac{1}{\tau_{elec}})-\frac{\omega^2\tau_0L}{\mathcal{L}R_0}\right)^{-1}
\end{equation}
Where $R_0$ is the resistance on the transition where the bolometer is initially biased to, $R_L$ is the shunt resistor in parallel with the TES, $\tau_{elec}$ is the electrical time constant, $I_0$ is the required current to initially set the TES at $R_0$, and $L$ is the inductance in the TES bias/readout circuit, and $\beta_I$ is the logarithmic derivative of TES resistance with current. The limits that we typically take for our experiments is $R_L/R_0 \ll 1$, $\beta_I \ll 1$, $\tau{elec} \ll \tau_{ETF}$, and $\tau{elec} \ll \tau_0$. Taking this limit from (1), $\tau_{ETF} \simeq \frac{\tau_0}{\mathcal{L}+1}$, and $\tau_{elec} = \frac{L}{R_L+R_0}$. Equation (2) also reduces  significantly to (3)
\begin{equation}
s_I(\omega) \simeq \frac{-1}{V_{biax}}\left(\frac{\mathcal{L}}{\mathcal{L}+1}\right)\left(\frac{1}{1-i\omega\tau_{ETF}}\right)\left(\frac{1}{i\omega\tau_{elec}-1}\right)
\end{equation}
Since $\tau{elec} \ll \tau_0$, we often just consider a narrow frequency band around the carrier tone and ignore the second pole from the electric transfer function so equation (3) can be written as (4).
\begin{equation}
s_I(\omega) \simeq \frac{-1}{V_{biax}}\left(\frac{\mathcal{L}}{\mathcal{L}+1}\right)\left(\frac{1}{1-i\omega\tau_{ETF}}\right)
\end{equation}
To account for mean optical power drifts in time periods between rebiasing of the bolometer we must either come up with a model for how mean optical power is changing on this time scale or find a way to actively measure it so that we can calibrate out effects on responsivity and time-constants. MS-F is working with NG-W to estimate how much we expect time constants \& sensitivity to vary over a PB1 scan from atmospheric loading data \& will add this number \& analysis here when complete.

A possible procedure to correct for time variation over the duration of a science scan is to interpolate our gains between measurements of the thermal calibration source taken at the beginning and end of the observation periods.
Obviously this technique is not perfect and will lead to some residual.

\paragraph{Plan to model and/or measure:}
In order to understand the impact of potential errors in this interpolation, we can for example construct a set of gains based only on the initial (or final) calibration measurement thus use no interpolation.
Say now that a simulated map with no $B$-modes is ``observed,'' producing timestreams using the non-interpolated gain model, and then reconstructed using the interpolated analysis gain model. 
The level of resulting \clbb\ (null to start with) quantifies the difference in these gain models in power spectrum space, and thus puts an upper limit on the impact of the drifts.

For Spider TES camera, and maybe Keck/BICEP I'm not sure, we would inject small electrical signals into the detectors regularly. We called this Bias Steps, it's written up in peoples' theses, but they might be embargoed, but if there's interest I'm sure that bit can be released. This, combined with lab measurements proving that it correlated well with actual gain, was very convienient.

For the NIKA LEKID camera, the NIKEL readout electronics monitored both on and near-resonance frequencies. That way when the DC loading, and therefore the gain, would change, this was monitored in realtime. Every single TOD sample had a corresponding gain measurement, which was nice because there didn't need to be interpolation. If SO uses LEKIDs, we should implement something like this. 

For DfMux we cannot do the same sort of correction that NIKA is doing with their KIDs because we are not tracking fractional frequency changes but instead tracking amplitude modulation. A similar scheme to NIKA may be thought up for mumux though. The bias steps method would be possibly by making small steps in the carrier amplitudes or a bias "tickle" procedure has been considered where you send a small tickle tone down on a lower side band of the carrier within the detector thermal bandwidth and then demodulate on the produced upper sideband. The variations in the demodulated sideband amplitude would correspond to changes in time constants from the behavior described above and could be related to a responsivity variation. This technique is currently in R\&D \& has not been deployed on PB1 (but it may have be tried on SPT-3G).

Bias Steps (described by Sean) are currently used in ACTPol for calibration. They can be performed quite often and do not perturb the detectors much (limited loss of data). Nevertheless, they do not provide an accurate responsivity for all detectors, so we identify a set of detectors for which the BS are accurate to set the gain, and flat-field the other detectors using the atmospheric signal.

Atmospheric signal is strongly dominant at low frequency and provide a real time relative calibrator. The El Nod calibration method relies on this strong signal, but difficulties arise from its modeling (at 1st order, the atmosphere is seen as homogeneous by the array; substructures should be taken into account to improve accuracy of the method though) and from the degeneracy between the atmosphere signal and thermal fluctuations in the cryostat.


\paragraph{Uncertainty/Range:}
\textbf{To be filled}
\paragraph{Parameterization:}
\textbf{To be filled}

%\paragraph{References}
%\begin{itemize}
%\item Irwin + Hilton
%\end{itemize} 

