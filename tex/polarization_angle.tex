\subsection{Polarization Angle}

\paragraph{Description:}
% Description of systematic effect, including relevant equations and
% parameterization for TWGs. Note that each variable in each equation should be
% defined. This should include where we expect to get the value of this variable
% from (TWG, literature, etc.)

Sources of polarization angle systematics are varied and can be introduced
several places in the instrument. To name a few examples, 1) a rotating elliptical beam (say
in the case of a design incorporating bore-sight rotations) can cause T to P
leakage, 2) off axis refractive optics influence the propagation of the
polarization vectors according to their Fresnel coefficients leading to an
instrumental polarization angle rotation and 3) in the presence of a HWP an
apparent polarization angle rotation can arise from the detector time
constants. 

Perhaps 1) and 3) are best left to their respective sections on beams and time
constants. Here we focus on 2), namely instrumental polarization errors and
detector polarization angle rotations. Understanding this instrumental
polarization error has implications not only for measuring CMB polarization,
but for placing constraints on Cosmic Polarization Rotation (CPR).

A global polarization rotation is degenerate with a CPR angle and affects the
power spectra as described in \cite{2013ApJ...762L..23K}. Analytic description
of instrumental rotation is challenging, necessitating the use of optical
modeling and experimental techniques for calibration of final detector angles
(absolute and relative) and systematic rotations from the optics.

\paragraph{Plan to model and/or measure:}
%Plan to model/measure effect
We should plan to both model and measure the detector polarization angles.
Calibration should be performed before deployment and during observations. This
can be done with a lab calibrator as well as an astrophysical polarized
source. Polarized sources already used as references include Tau A (the Crab
Nebula) and Cen A.

When considering a lab source, placing a well known polarization
calibrator in the far field is preferred, though difficult in practice.
Proposed ideas include artificial sources \cite{nati_2017} (whether tunable or wide band) flying on a
drone or on balloon or a CalSat to place them in the far field. Alternative calibrators
require placement in the near field and include sparse/dense wire grid
polarizers or dielectric sheets \cite{Takahashi2010, 2016arXiv160701825K}.

Modeling of the polarization rotation angle appears feasible and has been used
on ACTPol using Code V \cite{2016arXiv160701825K}. This should be checked with
physical optics calculations, but can be performed on a proposed telescope
design.

\paragraph{Uncertainty/Range:}
%This section should include the uncertainty of
%known parameters and/or the expected range of parameters for consideration

Angle offsets $\sim 1^{\circ}$ produce spurious B-mode signal at the same level
as primodial B-modes for a tensor to scalar ratio $r \sim 0.005$ as well as
nonzero $EB$ and $TB$ cross-correlations \cite{doi:10.1142/S0218271816400125}.
Currently employed calibration methods provide calibration to at best
$0.5^{\circ}$ \cite{2016MNRAS.455.1981K}. Calibration to better than
$0.05^{\circ}$ would allow for constraints on CPR of order one degree to
greater than $15\sigma$ \cite{2016MNRAS.455.1981K}.

\paragraph{Parameterization:}
%This section should include the parameterization of figures of
%merit and the output to the SWGs.
