
\subsection{HWP Beam Systematics}

Started By Sean Bryan

\paragraph{Description:}
Putting a HWP skyward of every optical element cures a range of beam and polarization systematics. However, putting a HWP inside the optical system induces some beam and polarization systematics of its own, and fails to cure some systematics you'd think it would. Thus, careful consideration is needed.

\paragraph{Plan to model and/or measure:}
Modeling would entail ray tracing and physical optics for a non-planar or non-uniform AR coated HWP. It would also require calculating the impact of non-uniform AR coatings or optical properties. Non-normal incidence effects...might matter...might focus the beam...ask ABS people on the theory side, and PB/ACT people on the experience side.

Also model how these beam systematics show up in the rapid-rotation demodulation. They will show up likely as I-P and Q-U leakage beams with very funny structure.

Measurements, in principle, would be done with a long campaign of beam maps taken at a large set of HWP angle. And/or, in the case of a rapid modulator, very slow beam maps taken with several modulation cycles per beam scanned (unlike in science mode where we can scan fast). The source is TBD, a sky source seems like it would be slow, and informally in the meeting people were saying that a source on a mast will be difficult because it requires tilting the telescope to a low elevation that's mechanically difficult.

\paragraph{Uncertainty/Range:}
Very uncertain "prior" on this, could be no effect, could be huge and a showstopper.

\paragraph{Parameterization:}

Leverage existing beam formalisms? Leverage Maria, Sean, Tom E-H, others, work on polarization non-idealities of the HWP, but combine it with beam formalisms?
