\subsection{Polarizatoin Angle Frequency Dependence of HWP}

\paragraph{Description:}
%Description of systematic effect, including relevant equations and
%parameterization for TWGs. Note that each variable in each equation should be
%defined. This should include where we expect to get the value of this variable
%from (TWG, literature, etc.)

Let's define polarization angle as the angle betweeen the light and the HWP, with incident light being rotated by the HWP to a different, transmitted angle.  This rotation angle has a dependence on the freqeuncy of light, as well as the precise properties of the HWP

It should be noted that there is a paper which proposes a way to mitigate this process\cite{Matsumura14}.  It places an additional, stationary HWP behind the rotating one.  This counteracts the frequency polarization angle dependence while maintianing broadband polarization modulation. 

\paragraph{Plan to model and/or measure:}
%Plan to model/measure effect. Use TRLs to describe how well we understand/can model the effect.
A Muller Matrix paramaterization should be sufficient to model this effect, although it also could be modeled using a full transmission line model.  The transmission line model may more accuratly capture the behavior since it takes the multiple relfections between the various layers into account.

For a silicon metamaterial HWP, the HWP in its entirity can be modeled in HFSS, and this dependence can be derived from the simulation.

The collaboration has several HWPs that can be used in lab to measure this effect, such as by using a vector network analyzer set up.

\paragraph{Uncertainty/Range:}
%This section should include the uncertainty of
%known parameters and/or the expected range of parameters for consideration
Discussion in \cite{Matsumura09} show this can be up to about 20 degrees for some HWPs.




\paragraph{Parameterization:}
%This section should include the parameterization of figures of
%merit and the output to the SWGs.

As discussed in \cite{Matsumura09}, a Muller Matrix representation of the HWP stack can give the frequency dependence of the polarization angle effectively.


