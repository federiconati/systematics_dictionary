\subsection{Degredation Over Time}

\paragraph{Description} 
% Description of systematic effect, including relevant equations and
% parameterization for TWGs. Note that each variable in each equation should be
% defined. This should include where we expect to get the value of this variable
% from (TWG, literature, etc.)

One concern for the HWP is degredation of the modulator performance over time. A few causes for such degredation are

\begin{enumerate}
	\item AR coating delamination
	\item AR coating degredation, for instance due to UV-induced deterioration
	\item Birefringent substrate material degredation, for instance due to cracking
	\item Birefringent substrate moving around inside the rotor, therefore adjusting its angle with respect to the encoder
	\item HWP becoming dirtied by the environment
\end{enumerate}

POLARBEAR is now in its third season using a modulator. The PB HWP consists of a sapphire sustrate with a layer of RT6002 Duroid from Rogers Corporation glued using melted LDPE for AR. The HWP is located between the primary and secondary mirrors (i.e. at prime focus) and is covered with a thin Mylar sheet for environmental protection. PB has not seen evidence of HWP degrdation despite snowstorms, windy weather, etc. The Mylar sheet should ideally be cleaned periodically.

For the POLARBEAR-2a warm HWP (WHWP), we are using a three-stack sapphire HWP with RT3006 Duroid/HDPE dual-layer AR held onto the sapphire using a vacuum-bag technique (i.e. no glue layers). The HWP sits in front of the optics tube window and is protected from the environment by a weather-sealed boom enclosure. The vacuum technique avoids the possibilities of AR delamination and non-uniform glue-layers. To prevent the three sapphire plates from rotating with respect to one another, we epoxy them together at the edges, and to prevent the stack from shifting in the rotor, it is held snugly by a rubber gasket as well as the friction of the vacuum bag. Another concern for HWP degredation is UV eroding the HDPE quality, but we expect this effect to be small given the UV-tight quality of the boom enclosure.

For the POLARBEAR-2b/c cold HWP (CHWP), we are using three-stack sapphire with alumina thermal spray loaded with nitrogen-infused silica microspheres to adjust the dielectric constant. The HWP will sit behind the cryostat window at the 50 K stage. An additional concern for the CHWP is the danger of AR delamination due to differential thermal contraction during thermal cycles. However, the alumina-powder-based thermal-spray AR has the same expansion coefficient as the sapphire substrate, and no delamination has been seen in testing.
  
\paragraph{Plan to model and/or measure:}
%Plan to model/measure effect

For the WHWP, because it is outside the receiver, load curves can be taken by doing in-and-out tests between seasons to look for uniform degredations of the dielectric layers that would cause an increase in emission. 

Non-uniform degredations, such as AR delamination, will show up in the HWP-synchronous-signals, or if egregiuos enough, in beam maps. These effects, though, depend on where your HWP is located within the optical system.

\paragraph{Uncertainty/Range:}
%This section should include the uncertainty of
%known parameters and/or the expected range of parameters for consideration

The acceptable range of these effects are difficult to quantify, but we should not tolerate AR delamination or shifting of the birefringent substrate in the design process. If it happens in the field, we must hope it's dramatic enough that we see it immediately.

\paragraph{Systematic Risk Factor}

These effects are important, but are easy to control for in design.

Risk: 2