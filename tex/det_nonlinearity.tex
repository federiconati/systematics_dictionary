\subsection{Detector Nonlinearity}
\label{det_nonlinearity}

\paragraph{Description:}
Assuming transition-edge sensors:

Inherent detector nonlinearity is masked by the formalism of Irwin & Hilton \cite{IrwinHilton:2008}. Since a detailed picture of the transition is not required to operate the TES in the strong feedback limit, we can treat the small-signal, strong-feedback case with explicitly linear equations. The strongly nonlinear detector response to changing temperature and current is parametrized with the parameters $\alpha$ and $\beta$.

However, it is well-known that the small-signal expansion around a ``bias point" (labeled with percent of normal resistance, bias current, and/or bath temperature) can be invalid in the field. Low-frequency modulation of the detector position in the transition (i.e. the bias point) can be caused by atmospheric or instrumental loading changes predominantly, and to a lesser extent by fluctuations in the cryogenic bath. This is the focus of the systematics entry on \ref{gain_drift}. In some sense, gain drift refers to the walk through the detector transition during observation and the multiple bias points around which the detector can be treated as linear.

Nonlinear response to large signals is a different topic, where the driving signal may swamp the feedback effect and drive the detector to different points on the transition \cite{Rostem}. Careful study of this effect could allow handling failures of the linear-gain, small-signal model. Doing so on a per detector-observation basis would be excessive. However, understanding nonlinear response may improve noise models used in maximum-likelihood mapmaking, analysis of calibrating signals from planets, and handling of modulator signals (i.e. improving demodulated performance at low frequency).

\paragraph{Plan to model and/or measure:}

Should be understood in terms of basic TES ($\alpha$, $\beta$) and bias condition ($P_o$, $I_o$) parameters, with a question about whether additional parameters are needed to handle the clipping effects of large-signal input. Frequency dependence is also an important factor, even in the definition of the problem.

\paragraph{Uncertainty/Range:}

\paragraph{Parameterization:}