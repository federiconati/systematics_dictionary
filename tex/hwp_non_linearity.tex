\subsection{Non-Linearity in Presence of HWP}

\paragraph{Description:} 
% Description of systematic effect, including relevant equations and
% parameterization for TWGs. Note that each variable in each equation should be
% defined. This should include where we expect to get the value of this variable
% from (TWG, literature, etc.)

The presence of a large HWPSS (usually at 2f or 4f), of order a few hundred mK, can drive the detectors in a non-linear regime. The detector response then changes synchronously with the HWP rotation harmonics. In particular, the response change induced by a large HWPSS 4f will modulate the unpolarized power at 4f, which appears in the detectors as an I $\rightarrow$ P leakage. This leakage can be partly removed by correlating $I$ and $P$ (done in the time domain for PB, see \cite{PB1_WHWP}, or in the map domain for EBEX, see \cite{joy_thesis_2016}), however the different sources of $I$ for the optical vs NL leakage (see $\delta_{opt}$), as well as gain variations $\epsilon$ and 1/f noise $\delta_{elec}$, couple into the removal procedure and produce a leftover $I \rightarrow P$ leakage.

\paragraph{Parameterization:}
We model the detector response in the following manner:

\begin{align}
d(t) = f_{nl} \left[  (I+\delta_{opt}) + (A_2 + a_2 I) e^{2 i \chi} + (Q + iU + A_4 + a_4 I)  e^{4 i \chi} \right] + \delta_{elec} \\
f_{nl}(x) = \left[ 1 + \epsilon + g_1 d(t - \tau_0 - \tau_1 I) \right] d(t - \tau_0 - \tau_1 I)
\end{align}

\noindent where we only included the 2f and 4f harmonics for $A(\chi)$. $I, Q$ and $U$ represent the CMB, foregrounds and atmosphere signals. $\delta_{opt}$ is the unpolarized power emitted after the HWP, $\delta_{elec}$ is the readout and detector 1/f noise. The non-linearity response is described by $f_{nl}$. (1 + $\epsilon$) is the gain, where $\epsilon$ is the error / time variation of the gain, $g_1$ is the variation of the gain with incoming power, $\tau_0$ is the time constant of the detectors, $\tau_1$ is the variation of the time constant with incoming power. \cite{PB1_WHWP} shows that $\tau_1$ and $g_1$ are linked to detectors parameters in the following way:

\begin{align}
g_1 = -\frac{\eta}{2 P_{elec}} \frac{L}{L+1} \frac{L+1+\omega_{mod}^2 \tau_0^2}{(L+1)^2 + \omega_{mod}^2 \tau_0^2} C \\
\tau_1 = \tau_0 \frac{\eta}{P_{elec}} \frac{L^2}{L+1} \frac{1}{(L+1)^2 + \omega_{mod}^2 \tau_0^2} C
\end{align}

\noindent The description of the parameters that go into $g_1$ and $\tau_1$, and usual values / varation with loop gain $L$ are available \href{http://simonsobservatory.wdfiles.com/local--files/calandsys-telecon/hwp_systematics_pipeline_2017-05-17.pdf?ukey=b7162749d5391f6bfc1b0d7e0ed84ab97c96f6a8}{here}
  
\paragraph{Plan to model and/or measure:}
%Plan to model/measure effect
A simulation implementing the model above has been written up by Neil, and Julien is currently implementing this model in the overall systematics simulator. The idea is to look for the induced P 1/f noise, and the effect on B-mode measurements after the $I \rightarrow P$ removal. A page with various details and documentation is on the wiki \href{http://simonsobservatory.wikidot.com/estimationof-hwpss}{here}.


\paragraph{Uncertainty/Range:}
The $I \rightarrow P$ leakage coming from non-linearity in PB $\sim$ 0.6\%. Given the HWPSS estimated magnitude, and worse/best case scenarios for the detector non-linearity parameters ($g_1$ and $\tau_1$, as described in the linked page), the $I \rightarrow P$ leakage coming varies from 0.13\% (low non-linearity, number driven by optical leakage) to 0.3 \% (high non-linearity). The residual $f_{knee}$ after removing the leakage is driven by the magnitude of $\epsilon$ (and to a lesser extent $\delta_{elec}$). Neil's \href{http://simonsobservatory.wdfiles.com/local--files/estimationof-hwpss/Estimating_Ell_Knee_for_SOLA.pdf?ukey=886f3c1ad56d36de42412c02765a1a658a35fad9}{simulations} show the leftover $f_{knee}$ is 0.7 Hz for $\epsilon$ variations on order of PB and 0.2 Hz for $\epsilon$ variations 1/10th of this.


